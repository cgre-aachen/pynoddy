% Generated by Sphinx.
\def\sphinxdocclass{report}
\documentclass[letterpaper,10pt,english]{sphinxmanual}
\usepackage[utf8]{inputenc}
\DeclareUnicodeCharacter{00A0}{\nobreakspace}
\usepackage{cmap}
\usepackage[T1]{fontenc}
\usepackage{babel}
\usepackage{times}
\usepackage[Bjarne]{fncychap}
\usepackage{longtable}
\usepackage{sphinx}
\usepackage{multirow}


\title{pynoddy Documentation}
\date{March 24, 2014}
\release{}
\author{Florian Wellmann}
\newcommand{\sphinxlogo}{}
\renewcommand{\releasename}{Release}
\makeindex

\makeatletter
\def\PYG@reset{\let\PYG@it=\relax \let\PYG@bf=\relax%
    \let\PYG@ul=\relax \let\PYG@tc=\relax%
    \let\PYG@bc=\relax \let\PYG@ff=\relax}
\def\PYG@tok#1{\csname PYG@tok@#1\endcsname}
\def\PYG@toks#1+{\ifx\relax#1\empty\else%
    \PYG@tok{#1}\expandafter\PYG@toks\fi}
\def\PYG@do#1{\PYG@bc{\PYG@tc{\PYG@ul{%
    \PYG@it{\PYG@bf{\PYG@ff{#1}}}}}}}
\def\PYG#1#2{\PYG@reset\PYG@toks#1+\relax+\PYG@do{#2}}

\expandafter\def\csname PYG@tok@gd\endcsname{\def\PYG@tc##1{\textcolor[rgb]{0.63,0.00,0.00}{##1}}}
\expandafter\def\csname PYG@tok@gu\endcsname{\let\PYG@bf=\textbf\def\PYG@tc##1{\textcolor[rgb]{0.50,0.00,0.50}{##1}}}
\expandafter\def\csname PYG@tok@gt\endcsname{\def\PYG@tc##1{\textcolor[rgb]{0.00,0.27,0.87}{##1}}}
\expandafter\def\csname PYG@tok@gs\endcsname{\let\PYG@bf=\textbf}
\expandafter\def\csname PYG@tok@gr\endcsname{\def\PYG@tc##1{\textcolor[rgb]{1.00,0.00,0.00}{##1}}}
\expandafter\def\csname PYG@tok@cm\endcsname{\let\PYG@it=\textit\def\PYG@tc##1{\textcolor[rgb]{0.25,0.50,0.56}{##1}}}
\expandafter\def\csname PYG@tok@vg\endcsname{\def\PYG@tc##1{\textcolor[rgb]{0.73,0.38,0.84}{##1}}}
\expandafter\def\csname PYG@tok@m\endcsname{\def\PYG@tc##1{\textcolor[rgb]{0.13,0.50,0.31}{##1}}}
\expandafter\def\csname PYG@tok@mh\endcsname{\def\PYG@tc##1{\textcolor[rgb]{0.13,0.50,0.31}{##1}}}
\expandafter\def\csname PYG@tok@cs\endcsname{\def\PYG@tc##1{\textcolor[rgb]{0.25,0.50,0.56}{##1}}\def\PYG@bc##1{\setlength{\fboxsep}{0pt}\colorbox[rgb]{1.00,0.94,0.94}{\strut ##1}}}
\expandafter\def\csname PYG@tok@ge\endcsname{\let\PYG@it=\textit}
\expandafter\def\csname PYG@tok@vc\endcsname{\def\PYG@tc##1{\textcolor[rgb]{0.73,0.38,0.84}{##1}}}
\expandafter\def\csname PYG@tok@il\endcsname{\def\PYG@tc##1{\textcolor[rgb]{0.13,0.50,0.31}{##1}}}
\expandafter\def\csname PYG@tok@go\endcsname{\def\PYG@tc##1{\textcolor[rgb]{0.20,0.20,0.20}{##1}}}
\expandafter\def\csname PYG@tok@cp\endcsname{\def\PYG@tc##1{\textcolor[rgb]{0.00,0.44,0.13}{##1}}}
\expandafter\def\csname PYG@tok@gi\endcsname{\def\PYG@tc##1{\textcolor[rgb]{0.00,0.63,0.00}{##1}}}
\expandafter\def\csname PYG@tok@gh\endcsname{\let\PYG@bf=\textbf\def\PYG@tc##1{\textcolor[rgb]{0.00,0.00,0.50}{##1}}}
\expandafter\def\csname PYG@tok@ni\endcsname{\let\PYG@bf=\textbf\def\PYG@tc##1{\textcolor[rgb]{0.84,0.33,0.22}{##1}}}
\expandafter\def\csname PYG@tok@nl\endcsname{\let\PYG@bf=\textbf\def\PYG@tc##1{\textcolor[rgb]{0.00,0.13,0.44}{##1}}}
\expandafter\def\csname PYG@tok@nn\endcsname{\let\PYG@bf=\textbf\def\PYG@tc##1{\textcolor[rgb]{0.05,0.52,0.71}{##1}}}
\expandafter\def\csname PYG@tok@no\endcsname{\def\PYG@tc##1{\textcolor[rgb]{0.38,0.68,0.84}{##1}}}
\expandafter\def\csname PYG@tok@na\endcsname{\def\PYG@tc##1{\textcolor[rgb]{0.25,0.44,0.63}{##1}}}
\expandafter\def\csname PYG@tok@nb\endcsname{\def\PYG@tc##1{\textcolor[rgb]{0.00,0.44,0.13}{##1}}}
\expandafter\def\csname PYG@tok@nc\endcsname{\let\PYG@bf=\textbf\def\PYG@tc##1{\textcolor[rgb]{0.05,0.52,0.71}{##1}}}
\expandafter\def\csname PYG@tok@nd\endcsname{\let\PYG@bf=\textbf\def\PYG@tc##1{\textcolor[rgb]{0.33,0.33,0.33}{##1}}}
\expandafter\def\csname PYG@tok@ne\endcsname{\def\PYG@tc##1{\textcolor[rgb]{0.00,0.44,0.13}{##1}}}
\expandafter\def\csname PYG@tok@nf\endcsname{\def\PYG@tc##1{\textcolor[rgb]{0.02,0.16,0.49}{##1}}}
\expandafter\def\csname PYG@tok@si\endcsname{\let\PYG@it=\textit\def\PYG@tc##1{\textcolor[rgb]{0.44,0.63,0.82}{##1}}}
\expandafter\def\csname PYG@tok@s2\endcsname{\def\PYG@tc##1{\textcolor[rgb]{0.25,0.44,0.63}{##1}}}
\expandafter\def\csname PYG@tok@vi\endcsname{\def\PYG@tc##1{\textcolor[rgb]{0.73,0.38,0.84}{##1}}}
\expandafter\def\csname PYG@tok@nt\endcsname{\let\PYG@bf=\textbf\def\PYG@tc##1{\textcolor[rgb]{0.02,0.16,0.45}{##1}}}
\expandafter\def\csname PYG@tok@nv\endcsname{\def\PYG@tc##1{\textcolor[rgb]{0.73,0.38,0.84}{##1}}}
\expandafter\def\csname PYG@tok@s1\endcsname{\def\PYG@tc##1{\textcolor[rgb]{0.25,0.44,0.63}{##1}}}
\expandafter\def\csname PYG@tok@gp\endcsname{\let\PYG@bf=\textbf\def\PYG@tc##1{\textcolor[rgb]{0.78,0.36,0.04}{##1}}}
\expandafter\def\csname PYG@tok@sh\endcsname{\def\PYG@tc##1{\textcolor[rgb]{0.25,0.44,0.63}{##1}}}
\expandafter\def\csname PYG@tok@ow\endcsname{\let\PYG@bf=\textbf\def\PYG@tc##1{\textcolor[rgb]{0.00,0.44,0.13}{##1}}}
\expandafter\def\csname PYG@tok@sx\endcsname{\def\PYG@tc##1{\textcolor[rgb]{0.78,0.36,0.04}{##1}}}
\expandafter\def\csname PYG@tok@bp\endcsname{\def\PYG@tc##1{\textcolor[rgb]{0.00,0.44,0.13}{##1}}}
\expandafter\def\csname PYG@tok@c1\endcsname{\let\PYG@it=\textit\def\PYG@tc##1{\textcolor[rgb]{0.25,0.50,0.56}{##1}}}
\expandafter\def\csname PYG@tok@kc\endcsname{\let\PYG@bf=\textbf\def\PYG@tc##1{\textcolor[rgb]{0.00,0.44,0.13}{##1}}}
\expandafter\def\csname PYG@tok@c\endcsname{\let\PYG@it=\textit\def\PYG@tc##1{\textcolor[rgb]{0.25,0.50,0.56}{##1}}}
\expandafter\def\csname PYG@tok@mf\endcsname{\def\PYG@tc##1{\textcolor[rgb]{0.13,0.50,0.31}{##1}}}
\expandafter\def\csname PYG@tok@err\endcsname{\def\PYG@bc##1{\setlength{\fboxsep}{0pt}\fcolorbox[rgb]{1.00,0.00,0.00}{1,1,1}{\strut ##1}}}
\expandafter\def\csname PYG@tok@kd\endcsname{\let\PYG@bf=\textbf\def\PYG@tc##1{\textcolor[rgb]{0.00,0.44,0.13}{##1}}}
\expandafter\def\csname PYG@tok@ss\endcsname{\def\PYG@tc##1{\textcolor[rgb]{0.32,0.47,0.09}{##1}}}
\expandafter\def\csname PYG@tok@sr\endcsname{\def\PYG@tc##1{\textcolor[rgb]{0.14,0.33,0.53}{##1}}}
\expandafter\def\csname PYG@tok@mo\endcsname{\def\PYG@tc##1{\textcolor[rgb]{0.13,0.50,0.31}{##1}}}
\expandafter\def\csname PYG@tok@mi\endcsname{\def\PYG@tc##1{\textcolor[rgb]{0.13,0.50,0.31}{##1}}}
\expandafter\def\csname PYG@tok@kn\endcsname{\let\PYG@bf=\textbf\def\PYG@tc##1{\textcolor[rgb]{0.00,0.44,0.13}{##1}}}
\expandafter\def\csname PYG@tok@o\endcsname{\def\PYG@tc##1{\textcolor[rgb]{0.40,0.40,0.40}{##1}}}
\expandafter\def\csname PYG@tok@kr\endcsname{\let\PYG@bf=\textbf\def\PYG@tc##1{\textcolor[rgb]{0.00,0.44,0.13}{##1}}}
\expandafter\def\csname PYG@tok@s\endcsname{\def\PYG@tc##1{\textcolor[rgb]{0.25,0.44,0.63}{##1}}}
\expandafter\def\csname PYG@tok@kp\endcsname{\def\PYG@tc##1{\textcolor[rgb]{0.00,0.44,0.13}{##1}}}
\expandafter\def\csname PYG@tok@w\endcsname{\def\PYG@tc##1{\textcolor[rgb]{0.73,0.73,0.73}{##1}}}
\expandafter\def\csname PYG@tok@kt\endcsname{\def\PYG@tc##1{\textcolor[rgb]{0.56,0.13,0.00}{##1}}}
\expandafter\def\csname PYG@tok@sc\endcsname{\def\PYG@tc##1{\textcolor[rgb]{0.25,0.44,0.63}{##1}}}
\expandafter\def\csname PYG@tok@sb\endcsname{\def\PYG@tc##1{\textcolor[rgb]{0.25,0.44,0.63}{##1}}}
\expandafter\def\csname PYG@tok@k\endcsname{\let\PYG@bf=\textbf\def\PYG@tc##1{\textcolor[rgb]{0.00,0.44,0.13}{##1}}}
\expandafter\def\csname PYG@tok@se\endcsname{\let\PYG@bf=\textbf\def\PYG@tc##1{\textcolor[rgb]{0.25,0.44,0.63}{##1}}}
\expandafter\def\csname PYG@tok@sd\endcsname{\let\PYG@it=\textit\def\PYG@tc##1{\textcolor[rgb]{0.25,0.44,0.63}{##1}}}

\def\PYGZbs{\char`\\}
\def\PYGZus{\char`\_}
\def\PYGZob{\char`\{}
\def\PYGZcb{\char`\}}
\def\PYGZca{\char`\^}
\def\PYGZam{\char`\&}
\def\PYGZlt{\char`\<}
\def\PYGZgt{\char`\>}
\def\PYGZsh{\char`\#}
\def\PYGZpc{\char`\%}
\def\PYGZdl{\char`\$}
\def\PYGZhy{\char`\-}
\def\PYGZsq{\char`\'}
\def\PYGZdq{\char`\"}
\def\PYGZti{\char`\~}
% for compatibility with earlier versions
\def\PYGZat{@}
\def\PYGZlb{[}
\def\PYGZrb{]}
\makeatother

\begin{document}

\maketitle
\tableofcontents
\phantomsection\label{index::doc}


Contents:


\chapter{pynoddy package}
\label{pynoddy:pynoddy-package}\label{pynoddy:welcome-to-pynoddy-s-documentation}\label{pynoddy::doc}

\section{Submodules}
\label{pynoddy:submodules}

\section{pynoddy.history module}
\label{pynoddy:pynoddy-history-module}\label{pynoddy:module-pynoddy.history}\index{pynoddy.history (module)}
Noddy history file wrapper
Created on 24/03/2014

@author: Florian Wellmann
\index{NoddyHistory (class in pynoddy.history)}

\begin{fulllineitems}
\phantomsection\label{pynoddy:pynoddy.history.NoddyHistory}\pysiglinewithargsret{\strong{class }\code{pynoddy.history.}\bfcode{NoddyHistory}}{\emph{history}}{}
Class container for Noddy history files
\index{change\_cube\_size() (pynoddy.history.NoddyHistory method)}

\begin{fulllineitems}
\phantomsection\label{pynoddy:pynoddy.history.NoddyHistory.change_cube_size}\pysiglinewithargsret{\bfcode{change\_cube\_size}}{\emph{cube\_size}}{}
Change the model cube size (isotropic)
\begin{description}
\item[{\textbf{Arguments}:}] \leavevmode\begin{itemize}
\item {} 
\emph{cube\_size} = float : new model cube size

\end{itemize}

\end{description}

\end{fulllineitems}

\index{determine\_events() (pynoddy.history.NoddyHistory method)}

\begin{fulllineitems}
\phantomsection\label{pynoddy:pynoddy.history.NoddyHistory.determine_events}\pysiglinewithargsret{\bfcode{determine\_events}}{}{}
Determine events and save line numbers

..Note: Parsing of the history file is based on a fixed Noddy output order.
If this is, for some reason (e.g. in a changed version of Noddy) not the case, then
this parsing might fail!

\end{fulllineitems}

\index{load\_history() (pynoddy.history.NoddyHistory method)}

\begin{fulllineitems}
\phantomsection\label{pynoddy:pynoddy.history.NoddyHistory.load_history}\pysiglinewithargsret{\bfcode{load\_history}}{\emph{history}}{}
Load Noddy history
\begin{description}
\item[{\textbf{Arguments}:}] \leavevmode\begin{itemize}
\item {} 
\emph{history} = string : Name of Noddy history file

\end{itemize}

\end{description}

\end{fulllineitems}

\index{write\_history() (pynoddy.history.NoddyHistory method)}

\begin{fulllineitems}
\phantomsection\label{pynoddy:pynoddy.history.NoddyHistory.write_history}\pysiglinewithargsret{\bfcode{write\_history}}{\emph{filename}}{}
Write history to new file
\begin{description}
\item[{\textbf{Arguments}:}] \leavevmode\begin{itemize}
\item {} 
\emph{filename} = string : filename of new history file

\end{itemize}

\end{description}

NB: Just love it how easy it is to `write history' with Noddy ;-)

\end{fulllineitems}


\end{fulllineitems}



\section{pynoddy.output module}
\label{pynoddy:module-pynoddy.output}\label{pynoddy:pynoddy-output-module}\index{pynoddy.output (module)}
Noddy output file analysis
Created on 24/03/2014

@author: Florian Wellmann
\index{NoddyOutput (class in pynoddy.output)}

\begin{fulllineitems}
\phantomsection\label{pynoddy:pynoddy.output.NoddyOutput}\pysiglinewithargsret{\strong{class }\code{pynoddy.output.}\bfcode{NoddyOutput}}{\emph{output\_name}}{}
Class definition for Noddy output analysis
\index{export\_to\_vtk() (pynoddy.output.NoddyOutput method)}

\begin{fulllineitems}
\phantomsection\label{pynoddy:pynoddy.output.NoddyOutput.export_to_vtk}\pysiglinewithargsret{\bfcode{export\_to\_vtk}}{\emph{**kwds}}{}
Export model to VTK

Export the geology blocks to VTK for visualisation of the entire 3-D model in an
external VTK viewer, e.g. Paraview.

..Note:: Requires pyevtk, available for free on: \href{https://github.com/firedrakeproject/firedrake/tree/master/python/evtk}{https://github.com/firedrakeproject/firedrake/tree/master/python/evtk}
\begin{description}
\item[{\textbf{Optional keywords}:}] \leavevmode\begin{itemize}
\item {} 
\emph{vtk\_filename} = string : filename of VTK file (default: output\_name)

\end{itemize}

\end{description}

\end{fulllineitems}

\index{load\_geology() (pynoddy.output.NoddyOutput method)}

\begin{fulllineitems}
\phantomsection\label{pynoddy:pynoddy.output.NoddyOutput.load_geology}\pysiglinewithargsret{\bfcode{load\_geology}}{}{}
Load block geology ids from .g12 output file

\end{fulllineitems}

\index{load\_model\_info() (pynoddy.output.NoddyOutput method)}

\begin{fulllineitems}
\phantomsection\label{pynoddy:pynoddy.output.NoddyOutput.load_model_info}\pysiglinewithargsret{\bfcode{load\_model\_info}}{}{}
Load information about model discretisation from .g00 file

\end{fulllineitems}

\index{plot\_section() (pynoddy.output.NoddyOutput method)}

\begin{fulllineitems}
\phantomsection\label{pynoddy:pynoddy.output.NoddyOutput.plot_section}\pysiglinewithargsret{\bfcode{plot\_section}}{\emph{direction='y'}, \emph{position='center'}, \emph{**kwds}}{}
Create a section block through the model
\begin{description}
\item[{\textbf{Arguments}:}] \leavevmode\begin{itemize}
\item {} 
\emph{direction} = `x', `y', `z' : coordinate direction of section plot (default: `y')

\item {} \begin{description}
\item[{\emph{position} = int or `center'}] \leavevmode{[}cell position of section as integer value{]}
or identifier (default: `center')

\end{description}

\end{itemize}

\item[{\textbf{Optional Keywords}:}] \leavevmode\begin{itemize}
\item {} 
\emph{ax} = matplotlib.axis : append plot to axis (default: create new plot)

\item {} 
\emph{figsize} = (x,y) : matplotlib figsize

\item {} 
\emph{colorbar} = bool : plot colorbar (default: True)

\item {} 
\emph{title} = string : plot title

\item {} 
\emph{savefig} = bool : save figure to file (default: show directly on scren)

\item {} 
\emph{fig\_filename} = string : figure filename

\end{itemize}

\end{description}

\end{fulllineitems}


\end{fulllineitems}



\section{Module contents}
\label{pynoddy:module-contents}\label{pynoddy:module-pynoddy}\index{pynoddy (module)}
Package initialization file for pynoddy
\index{compute\_model() (in module pynoddy)}

\begin{fulllineitems}
\phantomsection\label{pynoddy:pynoddy.compute_model}\pysiglinewithargsret{\code{pynoddy.}\bfcode{compute\_model}}{\emph{history}, \emph{output\_name}}{}
\end{fulllineitems}



\chapter{Indices and tables}
\label{index:indices-and-tables}\begin{itemize}
\item {} 
\emph{genindex}

\item {} 
\emph{modindex}

\item {} 
\emph{search}

\end{itemize}


\renewcommand{\indexname}{Python Module Index}
\begin{theindex}
\def\bigletter#1{{\Large\sffamily#1}\nopagebreak\vspace{1mm}}
\bigletter{p}
\item {\texttt{pynoddy}}, \pageref{pynoddy:module-pynoddy}
\item {\texttt{pynoddy.history}}, \pageref{pynoddy:module-pynoddy.history}
\item {\texttt{pynoddy.output}}, \pageref{pynoddy:module-pynoddy.output}
\end{theindex}

\renewcommand{\indexname}{Index}
\printindex
\end{document}
